\documentclass[twocolumn,landscape,10pt,a4paper]{article}

% Author: Nate Murray (based on Steve Tayon's Clojure cheat sheet)
% Comments, errors, suggestions: nate@natemurray.com

% License
% Eclipse Public License v1.0
% http://opensource.org/licenses/eclipse-1.0.php

% Packages
\usepackage{fullpage}
\usepackage[utf8]{inputenc}
\usepackage[T1]{fontenc}
\usepackage{textcomp}
\usepackage[english]{babel}
\usepackage{tabularx}
\usepackage{enumerate}
\usepackage[table]{xcolor}

% Set column space
\setlength{\columnsep}{0.25em}

% Define colours
\definecolorset{hsb}{}{}{red,0,.4,0.95;orange,.1,.4,0.95;green,.25,.4,0.95;yellow,.15,.4,0.95}
\definecolorset{hsb}{}{}{blue,.55,.4,0.95;purple,.7,.4,0.95;pink,.8,.4,0.95;blue2,.58,.4,0.95}
\definecolorset{hsb}{}{}{magenta,.9,.4,0.95;green2,.29,.4,0.95}

% Redefine sections
\makeatletter
\renewcommand{\section}{\@startsection{section}{1}{0mm}
	{-1.7ex}{0.7ex}{\normalfont\large\bfseries}}
\renewcommand{\subsection}{\@startsection{subsection}{2}{0mm}
	{-1.7ex}{0.5ex}{\normalfont\normalsize\bfseries}}
\makeatother

% No section numbers
\setcounter{secnumdepth}{0}

% No indentation
\setlength{\parindent}{0em}

% No header and footer
\pagestyle{empty}


% A few shortcuts
\newcommand{\cmd}[1] {\texttt{\textbf{{#1}}}}
\newcommand{\cmdline}[1] {
	\begin{tabularx}{\hsize}{X}
			\texttt{\textbf{{#1}}}
	\end{tabularx}
}

\newcommand{\colouredbox}[2] {
	\colorbox{#1!40}{
		\begin{minipage}{0.95\linewidth}
			{
			\rowcolors[]{1}{#1!20}{#1!10}
			#2
			}
		\end{minipage}
	}
}

% redefine enumerate env for closer spacing
\renewenvironment{enumerate}%
  {\begin{list}{\arabic{enumi}.}%
     {\topsep=0in\itemsep=0in\parsep=0in\partopsep=0in\usecounter{enumi}}%
   }{\end{list}}

\begin{document}
\twocolumn[
  \begin{@twocolumnfalse}
\centerline{\Large{\textbf{Git Cheat Sheet}}}
\centerline{All of the below commands are to be prefixed with \texttt{git}.}
\vspace{1cm}
  \end{@twocolumnfalse}
]

\section{Basics}

\colouredbox{green}{
\subsection{Documentation}
\cmdline{}
\begin{tabularx}{\hsize}{lX}
\cmd{help [cmd]} & \textit{Shows the manual entry for \texttt{[cmd]}} \\
\end{tabularx}
}

\colouredbox{magenta}{
\subsection{Working with remotes}
\begin{tabularx}{\hsize}{lX}
\cmd{clone <url>} & \textit{Clones repository located at 'url'.} \\
\cmd{pull} & \textit{Creates a local copy.}  \\
\end{tabularx}

\subsection{Create a local repository}
\begin{tabularx}{\hsize}{lX}
\cmd{init} & \textit{Creates an empty repository in a specified directory} \\
\cmd{add .} & \textit{Replaces file at a directory with a given file} \\
\cmd{commit} & \textit{This "activates" \cmd{add}, add will not affect git repository until commit is run.} \\
\end{tabularx}

\subsection{Files}
\begin{tabularx}{\hsize}{lX}
\cmd{add <file>} & \textit{Replaces file at a directory with a given file} \\
\cmd{rm <file>} & \textit{Remove files} \\
\cmd{mv <file>} & \textit{Move / rename a file or a directory} \\
\end{tabularx}

\subsection{Committing}
\begin{tabularx}{\hsize}{lX}
\cmd{commit -S} & \textit{Add a Signed off by line by the committer of the repository, at the end of the commit log} \\
\cmd{commit -S -{}-amend} & \textit{Lets you edit the signed off, without having to create a new commit.} \\
\end{tabularx}
}


%\colouredbox{blue}{
%\subsection{Configuration}
%\begin{tabularx}{\hsize}{lX}
%\cmd{config --{}--global user.name \textbackslash}\\
%\cmd{  "John Doe"} \textit{sets the global user name to "John Doe"} \\
%Email:  & \cmd{config --{}--global user.email \textbackslash}\\
%        & \cmd{  "email@example.com"} \textit{sets the global user email to "email@example.com"}
%\end{tabularx}
%}

\colouredbox{green}{
\subsection{Browsing}
\begin{tabularx}{\hsize}{lX}
\cmd{status} & \textit{Shows the files that are staged, unstaged, and untracked} \\
\cmd{log} & \textit{Lets you view the commit logs} \\
\cmd{blame <file>} & \textit{Lets you view what was last modified in a file} \\
\cmd{diff} & \textit{Show changes between commits.}
\end{tabularx}
}
\newpage

\section{Advanced}
\colouredbox{yellow}{
\section{Branching and Merging}
\begin{tabularx}{\hsize}{lXr}
List:     & \cmd{branch} & \textit{Allows you to list branches} \\
Create:   & \cmd{checkout -b new-branch} & \textit{Allows you to create a new branches} \\
Delete:   & \cmd{branch -d new-branch} & \textit{Allows you to delete branches} \\
Checkout: & \cmd{checkout testing} & \textit{Allows you to switch branches} \\
Merge:    & \cmd{merge testing} & \textit{Join branches together.}
\end{tabularx}
}





\colouredbox{blue2}{
\section{Remotes}
\begin{tabularx}{\hsize}{lXr}
Clone: &     \cmd{clone <url>} & \textit{Copies a GIT repository.}  \\
List: &     \cmd{remote -v} & \textit{Allows you to list remotes.}  \\
Send: &     \cmd{push } & \textit{Update remote references}  \\
      &     \cmd{push origin master} & \textit{Update references at the commits from your local master branch to a remote branch.}  \\
Receive: &     \cmd{pull } & \textit{Integrate with another repository or branch.}  \\
      &     \cmd{pull origin master} & \textit{Integrate with the commits from your local master branch to a remote branch.}  \\
Add:  &     \cmd{remote add <name> <url>} & \textit{Adds a remote named <name> for the repository at <url>}  \\
Fetch:&     \cmd{fetch name} & \textit{Download objects and references from name.}  \\
Checkout:&  \cmd{checkout <name/master>} & \textit{Allows you to switch by <name/master>}  \\
\end{tabularx}
}
%Site used for documentation: https://git-scm.com/
\begin{flushright}
\footnotesize
\rule{0.7\linewidth}{0.25pt}
\verb!Adapted from Cheat Sheet by Nate Murray:          ! \\
\verb!! \\
\verb!$Revision: 1.1, $Date: August 31, 2010!\\
\verb!Nate Murray (nmurray@xcombinator.com)!
\verb!See also: http://book.git-scm.com/!
\verb!          http://git.or.cz/course/svn.html!
\verb!Based on Steve Taylon's Clojure Cheat Sheet!
\end{flushright}

\end{document}

